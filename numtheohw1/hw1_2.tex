\begin{questions}
    \question{
.
    }
    \begin{solution}
        The $n^{th}$ pentagonal number $p_n$ is given by $$p_{n}
        =p_{n-1} + 3n +1$$
        where the base case is given by $p_0=1$.
    \end{solution}

    \question{
.
    }
    \begin{solution}
        We see that $p_0 = 1$ and using the recursive definition, 
        the few other terms are given by:
        \begin{align*}
            p_1 &= 1 +3*1+1=2+3*1\\
            p_2 &= \left(1+3*1+1\right) +3*2+1
            =3+3*3\\
            p_3 &= \left(3+3*3\right)+3*3+1
            =4+3*6\\
            &\vdots\\
            p_n &= (n+1)+3*t_n\\
            \shortintertext{where $t_n$ is the nth triangular number.}
        \end{align*}
        So, we have
        $$p_n = \sum_{i=0}^n{1+3i}$$
    \end{solution}

    \question{
        .
    }
    \begin{solution}
        We get, 
        \begin{equation}
            H_k: p_k = \sum_{i=0}^k{1+3i} = (k+1)+\frac{3k(k+1)}{2}
        =\frac{(k+1)(3k+2)}{2}
        \end{equation}

        Here, we see that the base case $p_0=2/2=1$, which is true. 
        Assuming that the statement $H_k$ is true for some $k$, we 
        have
        \begin{align*}
            p_{k+1} &= p_k+3(k+1)+1\\
            &=\frac{(k+1)(3k+2)}{2} +3k+4\\
            &=\frac{3k^2+5k+2+6k+8}{2}\\
            &=\frac{3k^2+11k+10}{2}\\
            &=\frac{(k+2)(3k+5)}{2}
        \end{align*}
        Then, since $H_k\implies H_{k+1}$, the statement is true for
        all integers $k\ge 0$.
        \begin{note}
            If we start the numbering of terms from 1, 
            we get the recursive formula $p_n=p_{n-1}+3(n-1)+1$ 
            and the closed formula 
            $$p_n=\frac{n(3n-1)}{2}$$
        \end{note}
    \end{solution}
\end{questions}