\begin{questions}
    \question{
        Bonus: Create a general formula for the $n$th
        polygonal number with sides of length $k$.
        For example $k=3$ for triangular numbers,
        $k=4$ for square numbers, $k=5$ for pentagonal
        numbers, etc...
        
        Use the notation $p(k,n)$ to mean the $n$th
        number on the list of the $k$-ogonal numbers.
        For example, $p(3,4) = 10$ because the 4th
        triagular number is 10.  While $p(4,3)=9$ since
        the 3rd square number is 9.
    }
    \begin{solution}
        Let $p(k,n)$ be the $nth$ number on the list of
        $k$-ogonal numbers. Then for all $k\ge3,
        k\in\mathbb{N}$, we have $p(k,1)=1$.

        To find the $(n+1)^{th}$ $k$-ogonal number, we 
        form $(k-2)$ sides, each of size $(n+1)$ and then 
        subtract $k-3$ (the number of common vertices)
        from the resultant. So we have
        $$p(k,n+1)=p(k,n)+(n+1)(k-2)+3-k$$
        We see that
        \begin{align*}
            p(k,1) &= 1\\
            p(k,2) &= 1+(2k-4)+3-k=k\\
            p(k,3) &= k+(3k-6)+3-k=3k-3\\
            p(k,4) &= 3k-3+(4k-8)+3-k=6k-8\\
            p(k,5) &= 6k-8+(5k-10)+3-k=10k-15\\
            &\vdots\\
            p(k,n) &= t_{n-1}k-s_{n-1} +1
        \end{align*}
        Thus
        \begin{align*}
            p(k,n) &= \frac{kn(n-1)}{2}-(n-1)^2+1\\
                   &= \frac{kn^2-kn-2n^2+4n-2+2}{2}\\
                   &= \frac{kn^2-2n^2-kn+4n}{2}\\
                   &= \frac{kn(n-1)}{2}+2n-n^2\\
        \end{align*}
        \qedsymbol
    \end{solution}
\end{questions}