
    \question{
        Bonus: Create a general formula for the $n$th
        polygonal number with sides of length $k$.
        For example $k=3$ for triangular numbers,
        $k=4$ for square numbers, $k=5$ for pentagonal
        numbers, etc...
        
        Use the notation $p(k,n)$ to mean the $n$th
        number on the list of the $k$-ogonal numbers.
        For example, $p(3,4) = 10$ because the 4th
        triagular number is 10.  While $p(4,3)=9$ since
        the 3rd square number is 9.
    }
    \begin{solution}
        Let $p(k,n)$ be the $nth$ number on the list of
        $k$-ogonal numbers. Then for all $k\ge3,
        k\in\mathbb{N}$, we have $p(k,1)=1$.

        To find the $(n+1)^{th}$ $k$-ogonal number, we 
        add $(k-2)$ sides, each of size $(n+1)$ and then 
        subtract $k-3$ (the number of common vertices)
        from the resultant. So we have
        $$p(k,n+1)=p(k,n)+(n+1)(k-2)+3-k$$
        Using this recursive definition, we find
        \begin{align*}
            p(k,n)&=\sum_{i=1}^n{
                    (i(k-2)+3-k)
            }\\
            &= \frac{n(n+1)}{2}(k-2)+n(3-k)\\
        \end{align*}
        \begin{note}
            We can 
            also see that $p(k,n) = kt_{n-1}-s_{n-1}+1$.
        \end{note}
    \end{solution}