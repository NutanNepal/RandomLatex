\question{
    It has been conjectured that every
    even number greater than 2 is the
    sum of two primes.  This is known
    as Goldbach’s conjecture, since
    he asked the question in 1742 in
    a letter to Euler.
}

\begin{parts}
    
\part{
    Write every even integer
    between 4 and 20, inclusively
    as the sum of two primes.
}
\begin{solution}
    \begin{align}
        4&=2+2,\\
        6&=3+3,\\
        8&=5+3,\\
        10&=5+5,\\
        12&=5+7,\\
        14&=7+7,\\
        16&=13+3,\\
        18&=11+7,\\
        20&=17+3.
    \end{align}
\end{solution}

\part{
    If you wanted to show that is
    holds for a large number, say
    1234567890, then describe
    a procedure to find such a sum.
    Yes, you may write it as a
    computer program if
    you so choose.
}
\begin{solution}
    We start from the half of the given
    number
    and check for each odd numbers.
    \begin{lstlisting}[language = Python]
#returns the primes
def goldbachify(n):
    [x, y] = [n//2,n//2]
    if not x % 2: [x,y] = [x + 1, y - 1]
    for _ in range (1, y):
        if is_prime(x) and is_prime(y):
            return [x, y]
        [x,y]=[x+2, y-2]
    return [x,y]
#check primality
def is_prime(n):
    if not n % 2: return False
    for x in range(3, int(n**(0.5))+1, 2):
        if not n % x: return False
    return True
    \end{lstlisting}
\end{solution}

\part{
    Give an example of an odd
    integer that cannot be written
    as the sum of two primes.
}
\begin{solution}
    23 cannot be represented as sum of
    two prime numbers.
\end{solution}

\part{
    Classify all odd numbers that
    cannot be expressed as the sum
    of two primes.
}
\begin{solution}
    Since odd numbers are always
    partitioned into an odd and even,
    and odd number $k$ can be written
    as $k=2m+(2n+1)$ for some natural
    numbers $m,n$. Since one of them
    is always even, $k$ cannot be
    written as sum of two primes if
    $k-2$ is not prime.
\end{solution}
\end{parts}